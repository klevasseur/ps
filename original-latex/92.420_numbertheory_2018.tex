%% AMS-LaTeX Created with the Wolfram Language : www.wolfram.com

\documentclass{article}
\usepackage{amsmath, amssymb, graphics, setspace}

\newcommand{\mathsym}[1]{{}}
\newcommand{\unicode}[1]{{}}

\newcounter{mathematicapage}
\begin{document}


\section{Number Theory}



Here are some basic number theory topics that are useful to know:


\item Definition (\pmb{ Congruence modulo }\pmb{ \textit{ n}}\pmb{ )}: { }\(a \equiv  b (\text{mod} n) \Longleftrightarrow  n \text{divides} \text{evenly}
\text{into} b-a\) \(\Longleftrightarrow\) \(b-a=n k\) { }for some integer \(k\).


\item Modular arithmetic { }(clock arithmetic with arbitrary moduli instead of just 12/24)


\item Greatest common divisors, least common multiples and relatively prime pairs, Euclid's algorithm. 


\item Wilson{'}s Theorem { }\((n-1)! +1\)is divisible by \(n\) if and only if { }\(n\) is prime.


\item Fermat{'}s Little Theorem and Euler's Theorem


\item Primitive Pythagorean Triples. { }One formula for all is { }\(\left(s t, \frac{s^2-t^2}{2}, \frac{s^2+t^2}{2}\right)\), where \textit{ s} and
\textit{ t} are odd integers, \(s>t\geq 1\), with no common factors greater than { }1.


\subsubsubsection{Problems to work on}

Prove that some positive integral power of 17 ends in 0001 (base 10).

Show, without using a calculator, that \(2^9 + 2^{99}\) is divisible by 100.

Let \(a, b\) be integers and \(m, n\) positive integers such that \(\gcd (m,n)=1\). { }Prove that there is a unique integer, \(x\), { }in the set
\(\{0,1,2,\ldots , m n -1\}\) such that \(x\equiv a (\text{mod} m)\) and \(y\equiv b (\text{mod} n)\).

Let \(\varphi (k)\) be the number of integers in the set \(\{1,2, \ldots  , k-1\}\) that are relatively prime to \(k\). Find and prove a formula
for \(\varphi \left(p^n\right)\) if \(n\geq 1\) and \textit{ p} is prime.

 For which positive integers \(n\) is there a sum of \(n\) consecutive integers that is a perfect square?

Show that the equation \(x^2- y^2=2x y z\) has no solutions in the positive integers. { }Hint: { }Consider a prime divisor of \(x y\).

Show that \(n^2+ 1\) is divisible by 7 for no positive integer \(n\).

Find all integral solutions to \(\left|p^r-q^s\right|=1\), where \textit{ p} and \textit{ q} are primes and \textit{ r} and \textit{ s} are positive
integers greater than or equal to 2. Prove that there are no other solutions than the ones you list.

What is the sum of the digits of the sum of the digits of the sum of the digits of \(4444^{4444}\)?

Each of the numbers \(x_1\), \(x_2\), ..., \(x_n\) equal 1 or -1, and { }\(x_1x_2x_3x_4+x_2x_3x_4x_5+x_3x_4x_5x_6+\cdots +x_{n-3}x_{n-2}x_{n-1}x_n+x_{n-2}x_{n-1}x_nx_1+x_{n-1}x_nx_1x_2+x_nx_1x_2x_3=
0\). { }\\
Prove that \(n\) is divisible by 4.

Find all positive integers \(n\) such that \(n!\) ends in exactly 1000 zeros.

Prove that among any three distinct integers we can find two, say \(a\) and \(b\), such that the number \(a^3b - a b^3\) is a multiple of 10.

\end{document}
