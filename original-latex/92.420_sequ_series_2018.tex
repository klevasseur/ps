%% AMS-LaTeX Created with the Wolfram Language : www.wolfram.com

\documentclass{article}
\usepackage{amsmath, amssymb, graphics, setspace}

\newcommand{\mathsym}[1]{{}}
\newcommand{\unicode}[1]{{}}

\begin{document}

\title{Sequences and Series}
\author{}
\date{}
\maketitle

\begin{doublespace}
\noindent\(\pmb{\text{}}\)
\end{doublespace}



The Fibonacci sequence: { } \(f_0=1,\text{  }f_1= 1\), and for \(n\geq 2\), \(f_n= f_{n-2}+f_{n-1}\).



Warm-up { }Prove that any two consecutive Fibonacci numbers are coprime. { }Note: { }the terms {``}coprime{''} { }and { }{``}relatively prime{''}
{ }mean the same thing, have no common factor other than 1.

 Consider products of pairs of Fibonacci numbers that are two positions apart in the sequence. { }What do you see? { }Describe the pattern and write
a general formula. { }Use your observation to examine the ratios of consecutive Fibonacci numbers. { } What happens in the {``}long run.{''}

 { }Find the sum { }\(1\cdot 1! + 2\cdot 2! + \cdots +n\cdot n!\).

 { }Evaluate { }\(\lim_{n\to \infty }  \left(\frac{1}{n+1}+ \frac{1}{n+2}+ \cdots +\frac{1}{2n}\right)\)

 { }\(\left\lfloor \sqrt{44}\right\rfloor =6\) and { } \(\left\lfloor \sqrt{4444}\right\rfloor =66\). { }Generalize and prove.

Express \(\sum _{k=1}^{\infty } \frac{6^k}{\left(3^{k+1}-2^{k+1}\right) \left(3^k-2^k\right)}\) as a rational number.

 Prove that \(\sum _{k=1}^n \frac{1}{\sqrt{k}}<2\sqrt{n}\)

Show that if the sequence \(\left\{a_n\right\}\) is monotonically decreasing and \(\sum _{n=1}^{\infty } a_n\) converges, then \(\sum _{n=1}^{\infty
} n \left(a_n-a_{n+1}\right)\) converges and the two sums are equal.

 Prove the following or provide a counterexample: { }If \(\left\{f_n\right\}\) is a sequence of functions defined on the interval \([0,1]\) and that
for each \textit{ x} in the interval \(\lim_{n\to \infty } f_n(x) = f(x)\), { }then { } { }\(\lim_{n\to \infty } \int_0^1 f_n(x) \, dx=\int_0^1 f(x)
\, dx\).

 { }The sequence \(a_0, a_1, a_2,\ldots\) satisfies\\
$\quad $\(a_{m+n}+a_{m-n}=\frac{1}{2}\left(a_{2m}+a_{2n}\right)\)\\
for all nonnegative integers \textit{ m} and \textit{ n} with \(m\geq n\). { }If \(a_1= 1\), determine \(a_n\).

 (a) { }How many ways can you tile a \(2 \times  n\) rectangle, \(n\geq 1\), using \(1\times 2\) tiles?\\
\hspace*{0.5ex} (b) How many ways can you tile a \(3 \times  2n\) rectangle, \(n\geq 1\), using \(1\times 2\) tiles?

Let \(p(x)=x^2-3x + 2\). { }Show that for any positive integer \(n\) there exist unique numbers \(a_n\) and \(b_n\) such that the polynomial \(q_n(x)=
x^n-a_nx-b_n\) is divisible by \(p(x)\). { } { }Hint: { }If true, then for all \(n\geq 1\), { } \(q_{n+1}(x)-x q_n(x)\) is divisible by \(p(x)\).

\end{document}
