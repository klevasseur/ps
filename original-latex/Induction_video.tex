%% AMS-LaTeX Created with the Wolfram Language : www.wolfram.com

\documentclass{article}
\usepackage{amsmath, amssymb, graphics, setspace}

\newcommand{\mathsym}[1]{{}}
\newcommand{\unicode}[1]{{}}

\newcounter{mathematicapage}
\begin{document}

\section*{Simple Induction}

\begin{doublespace}
\noindent\(\pmb{\text{Import}[\text{{``}/Users/ken$\_$l/Desktop/Screen Shot 2019-02-27 at 2.51.36 PM.png{''}}]}\)
\end{doublespace}

\begin{doublespace}
\noindent\(\)
\end{doublespace}

\pmb{ Proof by Induction:}\\
Basis: { } (\(n=2\)) { }Proven with a truth table as part of Section 3.5.\\
Induction: { }Assume the generalized DeMorgan{'}s Law with \(n\) propositions is true for some \(n\geq 2\). { } { }\\
$\quad $\(\neg \left(p_1\land p_2\land  \cdots \land p_n\land p_{n+1}\right) \Leftrightarrow \neg \left(\left(p_1\land p_2\land  \cdots \land p_n\right)\land
p_{n+1}\right)\text{  }\\
\\
\quad \quad \quad \quad \Leftrightarrow \neg \left(p_1\land p_2\land  \cdots \land p_n\right)\lor \left(\neg p_{n+1} \right)\text{     }\\
\\
\quad \quad \quad \quad \Leftrightarrow \left( \left( \neg p_1\right)\lor \left(\neg p_2\right)\lor  \cdots \lor \left(\neg p_n\right)\right)\lor
\left(\neg p_{n+1}\right)\\
\\
\quad \quad \quad \quad \Leftrightarrow  \left( \neg p_1\right)\lor \left(\neg p_2\right)\lor  \cdots \lor \left(\neg p_n\right)\lor \left(\neg p_{n+1}\right)\)

Therefore, it follows that the generalized DeMorgan{'}s Law with \(n+1\) propositions is true.

\section*{{``}Course of Values{''} Induction}

Here is an example of { }a connected undirected planar graph:

\begin{doublespace}
\noindent\(\pmb{\text{SeedRandom}[19];\text{PlanarGraph}[\text{Map}[\text{UndirectedEdge}\text{@@}\#\&,\text{Table}[\text{Sort}[\text{RandomSample}[\text{Range}[8],2]],\{15\}]\text{//}\text{Union}]]}\)
\end{doublespace}

\begin{doublespace}
\noindent\(\pmb{}\)
\end{doublespace}

\pmb{ Euler{'}s Formula: { }}Prove that if \(G\) is a connected undirected planar graph, with v vertices, e edges and r regions, then \(v+r-e=2\)

\(p(n)\): { }The statement is true for all such graphs with \(n\) edges, where \(n\geq 0\).

Basis: { }If a connected undirected planar graph has zero edges, it must be that the graph has only one vertex and the plane is {``}divided{''} into
only one region. { }Therefore \(v=1\), \(e=0\), and \(r=1\); and { } \(v+r-e=1+1-0 =2\).\\
\\
Induction. { }Assume Euler{'}s formula is true for all connected undirected planar graphs { }with \(k\) edges where \(k\leq n\). { } Assume we such
a graph, \(G\), with \(n+1\) edges. { }We remove one of the edges from \(G\), making it an undirected planar graph with \(n\) edges. { }However it
may no longer be connected. { }We need to consider two possible cases.

Case 1: { }The graph with the removed edge is still connected. In the example above, if we remove \(e_1\), we have this situation. { }Since we have
a connected undirected planar graph with \(n\) edges, the induction hypothesis applied with \(k=n\), and so Euler{'}s formula is true. { }Now, consider
returning the edge we had removed. { }In so doing, the number of edges increases by 1, the number of verticies stays the same, and the number of
regions increases by 1 since the returned edge divides a region into two parts. { }Therefore, the net change in the expression \(v+r-e\) is \(0+1-1=0\)and
so it is still equal to 2.\\
\\
Case 2: The graph with the removed edge has two components. In the example above, if we remove \(e_2\), we have this situation. { } The vertices
in each of the two components are connected. { }Assume the components have \(k_1\) and \(k_2\) edges, where \(k_1+k_2= n\). { }Furthermore, assume
the numbers of vertices and regions in the first component is \(v_1\) and \(r_1\), while the second component has \(v_2\) and \(r_2\) for it{'}s
vertices and regions. By the induction hyponthesis, \\
$\quad $\(v_1+r_1-k_1=2\) { } and { } \(v_2+r_2-k_2=2\)

$\quad $Now when we bring back the edge we had removed, the original graph had \(v_1+ v_2\) vertices and \(k_1+k_2+1= n+1\) edges. { }The number
of regions is \(r_1+ r_2-1\) because the infinite regions of the two components { }is now one region. { }Collecting this information, we have \\
$\quad \quad $\(v+r-e = \left(v_1+v_2\right)+\left(r_1+ r_2-1\right) -\left(k_1+k_2+1\right)\\
\\
\quad = \left(v_1+r_1-k_1\right)+ \left(v_2+r_2+k_2\right) -2\\
\\
\quad =2+2-2 = 2\)\\
Therefore Euler{'}s formula is true for this case.

\end{document}
