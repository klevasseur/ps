%% AMS-LaTeX Created with the Wolfram Language : www.wolfram.com

\documentclass{article}
\usepackage{amsmath, amssymb, graphics, setspace}

\newcommand{\mathsym}[1]{{}}
\newcommand{\unicode}[1]{{}}

\begin{document}

\title{Games}
\author{}
\date{}
\maketitle



Mathematical games are a common type of problem in the Putnam. { }The two most recent cases have been difficult, A5 in 2017 and B5 in 2014. { }The
problems on this sheet are easier, but might help you if a game appears this year.



In analyzing games, it is \pmb{ always} assumed that players use optimal strategies, i. e., they make rational moves in their best interests.

\pmb{ Nim.} This game starts with three piles of stones. { }The piles contain 5, 7, and 11 stones. { }Two players take turns. On each turn, a player
must remove at least one stone, and may remove any number of stones provided they all come from the same pile. The goal is to be the last player
to take a stone. What player has a definite winning strategy; i. e. will always win if he/she makes the right moves.

A line of squares is laid out starting with square 1, then square 2 to its right, square 3 further right, etc. { } The number of squares can be as
long as you like. Put some coins in some of the squares. { } For example. { } one on square 1, two on square 3, three on square 5, { }and one on
square 6. { } Take turns moving one coin to the left. { }There are no restrictions otherwise. { }You can jump onto or over other coins, or jump clear
off the line. You can have any number of coins on a square. { }Your aim is to be the person who makes the last move. { }

Two players play the following game. They start with two piles of candy, one with 21 and the other with 20 candies. At each turn, a player eats one
of the piles and splits the other into two piles. The first player who can{'}t make a legal move loses. Who will win, the first player or the second
player?

Consider the following game played with a deck of \(2n\) cards numbered from 1 to \(2n\). The deck is randomly shuffled and \(n\) cards are dealt
to each of two players. Beginning with A, the players take turns discarding one of their remaining cards and announcing its number. The game ends
as soon as the sum of the numbers on the discarded cards is divisible by \(2n + 1\). The last person to discard wins the game. Assuming optimal strategy
by both A and B, what is the probability that A wins?

 Alice and Bob play a game in which they take turns removing stones from a heap that initially has \(n\) stones. The number of stones removed at
each turn must be one less than a prime. { }The winner is the player who takes the last stone. { }Alice plays first. { }Prove that there are infinitely
many \(n\) such that Bob has a winning strategy.

\pmb{  { }}Alice and Barbara play a game with a pack of \(2n\) cards, on each of which is written a positive integer. The pack is shuffled and the
cards laid out in a row, with the numbers facing upwards. Alice starts, and the girls take turns to remove one card from either end of the row, until
Barbara picks up the final card. Each girl{'}s score is the sum of the numbers on her chosen cards at the end of the game.

In \pmb{ Determinant Tic-Tac-Toe}, Player 1 enters a 1 in an empty 3 $\times $ 3 matrix. Player 0 counters with a 0 in a vacant position, and play
continues in turn until the 3 $\times $ 3 matrix is completed with five 1{'}s and four 0{'}s. Player 0 wins if the determinant is 0 and player 1
wins otherwise. Assuming both players pursue optimal strategies, who will win and how?

A game is played as follows. The first player selects an interval [a, b]. The second player selects an interval \([c, d] \subset  [a, b]\). The first
player selects an interval inside \([c, d]\), and so on. The game goes on forever. The first player will win if the intersection of all segments
contains a rational number. Is he going to win?

\end{document}
