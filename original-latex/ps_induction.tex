%% AMS-LaTeX Created with the Wolfram Language : www.wolfram.com

\documentclass{article}
\usepackage{amsmath, amssymb, graphics, setspace}

\newcommand{\mathsym}[1]{{}}
\newcommand{\unicode}[1]{{}}

\newcounter{mathematicapage}
\begin{document}

\title{Induction and Pigeonholes}
\author{}
\date{}
\maketitle

\pmb{ Some advice.} \textit{  Work in groups. Try small cases. Do examples. Look for patterns. Use lots of paper. Talk it over. Choose effective
notation. Try the problem with different numbers. Work backwards. Argue by contradiction. Modify the problem. Generalize. Don{'}t give up after five
minutes. Don{'}t be afraid of a little algebra. Sleep on it. { }If need be, ask.}

\subsection*{Examples}

A. \pmb{ Fermat{'}s little theorem.} Let \textit{ p} be a prime number, and \textit{ n} a positive integer. Then \(n^p - n\)is divisible by \textit{
p}.

B. { }Prove that every set of 10 two-digit integer numbers has two disjoint subsets with the same sum of elements. { }(Subsets are pigeons, possible
sums are holes)

\subsection*{Problems to work on }



Find and prove a formula for the sum of the first \textit{ n} consecutive odd positive integers. For example, if \(n = 4\) then \(1 + 3 + 5 + 7 =
16\).

Prove that in a room with \(n\) people, at least two people know exactly the same number of people. Assume knowing is a symmetric relation: If Paul
knows Pat, then Pat knows Paul.

Let \(S\) be any set of 18 distinct integers chosen from the arithmetic progression \(1, 4, 7, . . . , 100\). { }Prove that there must be two integers
in \(S\) whose sum is 101.

Prove for all positive integers \(n\) the identity

 \(\frac{1}{n+1}+\frac{1}{n+2}+\cdots +\frac{1}{2n}=1-\frac{1}{2}+\frac{1}{3}-\frac{1}{4}+\cdots +\frac{1}{2n-1}-\frac{1}{2n}\).

Choose 51 positive integers from 1 to 100. Prove that one of them is a multiple of another.

Show that a \(2^n \times  2^n\) square with a corner tile removed can be covered without overlaps by L-shaped figures (each figure contains 3 tiles).
(If you feel adventurous, how about an \(n \times  n\) square for arbitrary \textit{ n}?)

Given nine points inside the unit square, prove that some three of them form a triangle whose area does not exceed 1/8.

Given a sequence of integers \(x_1\), \(x_2\), \(\text{...}x_n\) whose sum is 1, prove that exactly one of the cyclic shifts

 { }\(x_1,x_2,\text{...},x_n\)\\
 { }\(x_2,\text{...},x_n,x_1\)\\
\hspace*{3.5ex} $\vdots $\\
 { } { } { } \(x_n,x_1,\text{...},x_{n-1}\)

has all of its partial sums positive. (By a partial sum we mean the sum of the first\\
\(k\)terms, \(k \leq  n\).)

Show that there is a positive term of the Fibonacci sequence that is divisible by 1000. Recall that the Fibonacci { }sequence is defined by \(F_0=
F_1= 1\) and for \(n>1\), \(F_n= F_{n-1}+F_{n-2}\).

You have coins \(C_1\), \(C_2\), ... , \(C_n\). For each \textit{ k}, \(C_k\) is biased so that, when tossed, it has probability \(1/(2k + 1)\) of
falling heads. If the \(n\) coins are tossed, what is the probability that the number of heads is odd? Express the answer as a rational function
of \(n\).

Prove that any positive integer can be represented as \(\pm 1^2\pm 2^2\pm 3^2\pm \cdots \pm n^2\) for some positive integer \(n\) and some choice
of signs. { }Hint: { }Basis uses the first four integers, and then do four inductions.

\subsection*{Turn in next week: at least one of the following}

I. { }Prove that for any positive integer \(n\) there exists an \(n\)-digit number divisible by \(2^n\) and containing only the digits 2 and 3.

II. { }Inside a circle of radius 4 are chosen 61 points. Show that among them there are two at distance at most \(\sqrt{2}\) from each other.

If you can{'}t make progress on either of I or II, write out a detailed solution to one of the numbered problems between 4 and 11.



\end{document}
