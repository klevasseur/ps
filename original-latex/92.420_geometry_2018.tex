



There are three basic approaches to geometric problems:



(1) Axiomatic approach, where everything is deduced from basic facts, such as congruence tests for triangles (SAS, ASA, and SSS), similarity of triangles,
angles in the circle theorem, etc. These problems are great to sharpen your proof skills but you are unlikely to see them on the Putnam exam.



(2) Method of coordinates. Points on the plane are interpreted as coordinates <m>(x,y) \in \mathbb{R}^2</m>, or vectors, or complex numbers. Calculations
can often be simplified by using basic linear algebra (scalar products, etc.) and knowing geometric interpretations of various algebraic operations
(e.g. multiplication of complex numbers). Alternatively, a lot of things can be computed using trig functions.



(3) Symmetries and transformations. This is a more dynamic approach, where you apply and compose rotations, symmetries, etc.



Often, problems are only formulated using geometric language but the solution uses some counting trick, or combinatorics, etc.





Prove that a central angle subtended by a given circular arc is twice the angle of an inscribed angle for the same arc.

Inscribe a rectangle of base <m>b</m> and height <m>h</m> and an isosceles triangle of base <m>b</m> in a circle of radius one as shown. { }For what value
of <m>h</m> do the rectangle and triangle have the same area? 



What convex quadrilaterals can be inscribed into a circle? { }There is a name for these quadrilaterals, but your answer should describe them, not
just name them.

 Prove that if the lengths of the sides of a triangle form an arithmetic progression, then the radius of the inscribed circle is one third of one
of the heights of the triangle.

A piece of paper is in the shape of a rectangle <m>ABCD</m> with <m>AB=CD=3</m> and <m>AD=BC=5</m>.
{ }The paper is folded so that <m>A</m> and <m>C</m> coincide. { }Find the length of the crease. { } Generalize your result.

On the hyperbola <m>x y = 1</m> consider four points whose <m>x</m>-coordinates are <m>x_1</m>, <m>x_2</m>, <m>x_3</m>, and <m>x_4</m>. { }Show that if these points
lie on a circle, then <m>x_1\cdot  x_{2 }\cdot x_{3 }\cdot x_4 = 1</m>.

Let <m>ABC</m> be a triangle and let the angle bisector of <m>\text{$\angle $A}</m> intersect the side <m>BC</m> at a point <m>D</m>. Show
that <m>A B/A C = B D/C D</m>.

Let <m>f</m> be a real-valued function on the plane such that for every square <m>ABCD</m> in the plane, <m>f(A) + f(B) + f(C) + f(D) = 0</m>.
Does it follow that <m>f(P) = 0</m> for all points <m>P</m> in the plane?

Given nine lattice points in <m>\mathbb{R}^3</m> prove that there exist two of them with the property that the midpoint of the segment between them
is a lattice point.

Right triangle <m>\text{\textit{ABC}}</m> has a right angle at <m>C</m> and $\angle $<m>BAC</m> = $\theta $; the point <m>D</m> is chosen on <m>AB</m>
so that <m>\text{<m>AC</m>} = \text{<m>AD</m>}\text{\textit{$ $}}= 1</m>; the point <m>E</m> is chosen on <m>BC</m> so that $\angle $\textit{
CDE} = $\theta $. The perpendicular to <m>BC</m> at <m>E</m> meets <m>AB</m> at <m>F</m>. Evaluate { } <m>\lim _{\theta \to 0} \text{<m>EF</m>}.</m>

Consider triangle <m>ABC</m> with the following trisection points,\\
(1)<m>P</m> on segment <m>AB</m> closest to <m>B</m>\\
(2) <m>R</m> on segment <m>BC</m> closest to <m>C</m> and\\
(3) <m>Q</m> on segment <m>CA</m> closest to <m>A</m>.\\
If these points are connected by segments to the opposite vertices of the triangle, is the area of the inner triangle created by the segments related
to the area of triangle <m>ABC</m>?



<m>\pmb{12.}</m>


\section{Assignment Due Next Week}



Turn in the following:


\item Any two of the problems from 6 through 12


\item Pick any problem you haven{'}t been able to do at any time in this semester, but which you{'}ve seen a complete solution. { } Describe how
you approached the problem and why it wasn{'}t successful. { } Explain how you could have come up with the solution if you had thought about it differently.





\end{document}
